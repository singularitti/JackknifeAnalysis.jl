\section{Argon Molecular Dynamics}

We can now apply these statistical ideas to the results of your argon MD simulation. Run as
long a simulation as is practical and make measurements of the temperature, potential energy
and the time average of the virial, which is given by
%
\begin{equation}
    \sum_i \sum_{j > i} r_{ij} \frac{ \partial V_{ij} }{ \partial r_{ij} }
\end{equation}
%
every MD time step. You should be able to run a few thousand steps, after thermalization.


\Question{} Measure the autocorrelation times for the temperature, potential energy and
virial.

\Answer{}




\Question{} Also measure the covariance matrix for these \(3\) quantities.

\Answer{}



\Question{} Use your estimate of the autocorrelation times, along with binning and the
jackknife method to give an error on the pressure from your simulation.

\Answer{}

