\section{Argon Molecular Dynamics}

We can now apply these statistical ideas to the results of your argon MD simulation. Run as
long a simulation as is practical and make measurements of the temperature, potential energy
and the time average of the virial, which is given by
%
\begin{equation}
    \sum_i \sum_{j > i} r_{ij} \frac{ \partial V_{ij} }{ \partial r_{ij} }
\end{equation}
%
every MD time step. You should be able to run a few thousand steps, after thermalization.


\Question{p3q1} Measure the autocorrelation times for the temperature, potential energy and
virial.

\Answer{}
We run the Argon molecular dynamics from the last homework for a total \(24046\) steps,
and after \(15000\) steps we reached desired temperature (around \(1.0718\)).
Therefore, there are \(9047\) total steps for doing a statistical analysis.

We plot the potential energy, kinetic energy, and total energy in Figure~\ref{fig:MD_e_t},
which corresponds to the simulation time from \(t = 480\) to \(770\).

\begin{figure}[hb]
    \centering
    \includegraphics[width=0.8\textwidth]{e-t}
    \caption{The potential energy, kinetic energy, and total energy for the
        Argon molecular dynamics from simulation time from \(t = 480\) to \(770\).}
    \label{fig:MD_e_t}
\end{figure}

The temperature and virial terms are plotted in
Figure~\ref{fig:MD:a} and Figure~\ref{fig:MD:b}, respectively.

\begin{figure}[H]
    \centering
    \begin{minipage}[t]{0.8\linewidth}
        \centering
        \includegraphics[width=\linewidth]{MD_temperature}
        \subcaption{The temperature for the
            Argon molecular dynamics from simulation time from \(t = 480\) to \(770\).}
        \label{fig:MD:a}
    \end{minipage}
    \hfill
    \begin{minipage}[t]{0.8\linewidth}
        \centering
        \includegraphics[width=\linewidth]{MD_virial}
        \subcaption{The virial for the
            Argon molecular dynamics from simulation time from \(t = 480\) to \(770\).}
        \label{fig:MD:b}
    \end{minipage}
    \caption{The temperature and virial terms for the
        Argon molecular dynamics.}
    \label{fig:MD}
\end{figure}

Similar to Problem~\ref{sec:ja}, we plot the integrated autocorrelation time
for temperature, potential energy, and virial as functions of \(n_\textnormal{cut}\)
in Figures~\ref{fig:MD_iat:a},
\ref{fig:MD_iat:b}, and~\ref{fig:MD_iat:c}.
As we can see, they are all short compared to the total number of time steps,
this is because we have a relative small number of total steps, similar to
Problem~\ref{sec:mock}.

\begin{figure}
    \centering
    \includegraphics[width=0.8\textwidth]{MD_tau_virial}
    \caption{The integrated autocorrelation time for the virial.}
    \label{fig:MD_iat:a}
\end{figure}

\begin{figure}
    \centering
    \begin{minipage}[t]{0.8\linewidth}
        \centering
        \includegraphics[width=\linewidth]{MD_tau_temperature}
        \subcaption{The integrated autocorrelation time for temperature.}
        \label{fig:MD_iat:b}
    \end{minipage}
    \hfill
    \begin{minipage}[t]{0.8\linewidth}
        \centering
        \includegraphics[width=\linewidth]{MD_tau_energy}
        \subcaption{The integrated autocorrelation time for the potential energy.}
        \label{fig:MD_iat:c}
    \end{minipage}
    \caption{The integrated autocorrelation time for temperature and the potential energy.}
    \label{fig:MD_iat}
\end{figure}


\Question{} Also measure the covariance matrix for these \(3\) quantities.

\Answer{}
The covariance matrix for temperature, potential energy, and virial is
%
\begin{equation}
    \hat{c} =
    \begin{bmatrix}
        0.00024  & -0.30849  & 0.00425    \\
        -0.30849 & 698.43424 & 84.42649   \\
        0.00425  & 84.42649  & 1296.32289
    \end{bmatrix}.
\end{equation}
%
And the normalized version of the covariance is
%
\begin{equation}
    \hat{\rho} =
    \begin{bmatrix}
        1        & -0.75095 & 0.00759 \\
        -0.75095 & 1        & 0.08873 \\
        0.00759  & 0.08873  & 1
    \end{bmatrix}.
\end{equation}
%
As we can see, the temperature and potential energy are negatively correlated.
This is understandable since the total energy is fixed, and the potential energy and
the kinetic energy are negatively correlated, while the temperature and the kinetic
energy are positively correlated.
And the temperature and the virial are positive correlated since the kinetic energy
and the virial satisfy the following relation:
%
\begin{equation}
    \langle T \rangle = \frac{ 1 }{ 2 }
    \sum_i
    \biggl\langle r_i \frac{ \partial U }{ \partial r_i } \biggr\rangle.
\end{equation}


\Question{} Use your estimate of the autocorrelation times, along with binning and the
jackknife method to give an error on the pressure from your simulation.

\Answer{}
The size of each bin should be at least larger than the \(n_\textnormal{cut}\)
for each variable, as discussed in previously.
The expression of \(P\) has two related variables, i.e., temperature and virial:
%
\begin{equation}
    \frac{ \beta P }{ \rho } = 1 + \frac{ 1 }{ 3N_p T } \sum_{i=1}^{N}
    \langle \bm{r}_i \cdot \bm{F}_i \rangle,
\end{equation}
%
where \(N_p\) denotes the number of particles in the system.
Therefore, as in Problem~\ref{sec:ja} Question~\ref{p2q3}, we can split temperature
and virial into bins of size \(b\), where \(b\) should satisfy
%
\begin{equation}
    b \geq \max \{n_{\textnormal{cut}, \tau_T}, n_{\textnormal{cut}, \tau_v}\},
\end{equation}
%
where \(\tau_v\) denotes the integrated autocorrelation time for the virial.
We should select the same \(b\) for the two variables,
or else we do not have matching numbers of correlated data.

From Question~\ref{p3q1} we know that
\(n_{\textnormal{cut}, \tau_T} \geq 15\) and
\(n_{\textnormal{cut}, \tau_v} \geq 16\), so the \(b\) we will use should be
at least \(16\), but not too large since we do not want to include too much noises
into our samples. We will pick \(b = 19\) for simplicity.
However, we need to cut the last \(3\) data points to minimize bias.

Thus, we separate the two variables into \(N / b = 476\) groups, each with
size \(19\). Then we average the values in each group and treat it as one datum.
The function here is
%
\begin{equation}\label{eq:f}
    f(\bar{T}, \bar{v}) = 1 + \frac{ 1 }{ 3N_p \bar{T} } \bar{v},
\end{equation}
%
where \(\bar{v}\) denotes the average value of each virial term, since we have done
the summation over all particles when we calculte the virials, there is no need to
add that \(\sum\) symbol in Equation~\eqref{eq:f}.

After jackknife resampling, we have
%
\begin{equation}\label{eq:mdsigma}
    \sigma^2_{f} = \frac{ N/b - 1 }{ N/b }
    \sum_{k=1}^{N/b} \bigl( f(T'_k, v'_k) - f(\bar{T}, \bar{v}) \bigr)^2,
\end{equation}
%
where \(\bar{T}\) and \(\bar{v}\) denote the average values of temperature and
virial for all steps, and the summation is over each group of the jackknife
data.
After substituting data into Equation~\eqref{eq:mdsigma}, we obtain
\(\sigma_{f} \approx \num{4.08629e-4}\), which is a relative small error considering
the value of \(f\) is of magnitude around \(1\).
