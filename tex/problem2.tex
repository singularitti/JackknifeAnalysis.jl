\section{Jackknife Analysis}

In this problem, we will see how the jackknife method can be used to find errors on functions
on the mean values of data, i.e., on \(f(\bar{v}_a)\).
We can use the universe of data to calculate
the errors and then calculate the same error from a sample of size \(N = 5000\).

We will consider two functions in what follows
%
\begin{align}
    f_1(\bar{v}_1, \bar{v}_2) & = \frac{ \bar{v}_1 }{ \bar{v}_2 }, \\
    f_2(\bar{v}_3, \bar{v}_4) & = \exp( \bar{v}_3 - \bar{v}_4 ).
\end{align}

\Question{} Break the \(M\) measurements up into groups of size \(N\), calculate
\(\bar{v}_a\) for each group and then calculate \(f_i(\bar{v}_a)\) for each group.
Calculate these functions of the data means for all \(M/N\) groups and find the
standard deviation for \(f_i(\bar{v}_a)\), \(\sigma_{f_i,N}\).

\Answer{}





\Question{} Calculate \(\sigma_{f_i,N}\) from naive propagation of errors, i.e., using
\(\hat{\sigma}_{\bar{v}_a,N}\) and neglecting correlations between the \(v_i\). Compare with
your results from Problem~\ref{sec:mock}.

\Answer{}



\Question{} We now want to estimate \(\sigma_{\bar{v}_a,N}\) and \(\sigma_{f_i,N}\) using
the jackknife method from a single sample of size \(N\).
First we must deal with the autocorrelations in the data, and you have an idea of the
integrated autocorrelation time from the first problem. We proceed as follows here. Average
your \(N\) data values into bins of size \(b\). This will produce \(N/b\) data values.
Then use the jacknife method to estimate \(\bar{v}_a\) and \(\sigma_{\bar{v}_a,N}\) from
these \(N/b\) data values. The jacknife method resums these \(N/b\) values as done in class,
i.e.,
%
\begin{equation}
    v'_{a,k} = \frac{ 1 }{ N / b - 1 } \sum_{i=1, i \neq k}^{N/b} v_{a,i}.
\end{equation}

From these jacknife values, you can determine \(\bar{v}_a\) and \(\sigma_{\bar{v}_a,N}\).
Do this for a few different values of \(b\) comparable to the integrated autocorrelation
time to check that your results do not depend strongly on \(b\).

